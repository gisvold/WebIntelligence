\chapter{Videreutvikling}
\label{improvements}

Det er mye ved dette prosjektet som kunne vært gjort bedre. 

\section{Forbedringer i implementasjonen}
TF og TF-IDF er ikke egnet for å håndtere ulike dokumentformat som HTML, XML og .owl. Dette førte til at vi måtte gjøre en god del implementasjon for å gjøre preprosessering for å få dokumentene på en felles standard. I etterkant har vi lært at det eksisterer måter å gjøre dette på en mer egnet måte. Ved hjelp av 'Semantic-Sensitive Web Information Retrieval Model' kan man enklere hente ned dokumenter av ulikt format. %TODO: Legg til referanse

Vi valgte å skille kapitler i NLH på hovedkapitler og subkapitler. Dette førte til at resultatene våre ved enkelte tilfelle ble suboptimale. NLH deler videre inn i subsubkapitler, selv om disse dokumentene er deler av subkapitlene. Ved tolkningen og prosesseringen av NLH kunne vi delt dokumentene ytterligere. Vi gjorde et forsøk på å gjøre dette, men resultatene våre ble ustabile og dermed oppstod feil. Dette ble for tidkrevende for at vi kunne implementere det. 

\section{Forbedringer i arkitekturen}
Koden er generelt dårlig dokumentert og kommentert. Det vil være vanskelig for de som måtte ta over koden å arbeide videre på den. 


\section{Gruppesammensetning}
Gruppen vår bestod av tre studenter med tre ulike timeplaner. Dette førte til vanskeligheter med å jobbe sammen. Dette førte til økte komplikasjoner med kommunikasjonen i gruppen. Vi burde arrangert flere økter hvor vi jobbet sammen. 