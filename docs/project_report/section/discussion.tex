\section{Diskusjon}
\label{sec:diskusjon}

Det finnes flere viktige aspekter rundt informasjonsgjenfinning. Blant annet vil det ofte være naturlig å se på precision kontra recall. Som en del av kursmaterale fikk vi også med en såkalt 'gullstandard' som skulle brukes for å måle ytelsen av søkemotoren som vi har laget.

Precision går på det å se på antall relevante dokumenter av de dokumentene som er valgt. 

Precision = (relevante og hentede dokumenter) / hentede dokumenter. Eks. Om vi har hentet 2 dokumenter, og bare 1 er relevant. Så vil precision være 50\%.

Da ser vi mens recall ser på på antall relevanter dokumenter av totale dokumenter. Recall = (relevante og hentede dokumenter) / (totale dokumenter). Eks. om det finnes 20 dokumenter i kolleksjonen, og 10 av dem er hentet, så vil recall være 50\%. Precision ville vært 100\% i dette tilfelle. 



%TODO: the rest

\paragraph
To ensure a better picture of the situation, we also calculated the interpolated precision of four recall levels: 25\%, 50\%, 75\% and 100\%.

\begin{tabular}{|m{0.3 \textwidth}|m{0.3 \textwidth}|}
\hline
Recall & $\overline{P}$(r) \\ \hline
25\% & 68\% \\ \hline
50\% & 46\% \\ \hline
75\% & 28\% \\ \hline
100\% & 25\% \\ \hline
\captionof{table}{Precision vs Recall}
\end{tabular}

\begin{center}
\importgraphics{recallgraf}
\captionof{figure}{Precision vs Recall}
\end{center}



Since the NLH is quite a huge document, parted into a lot of chapters and sub-chapters, we decided only to consider subchapters that were given their own HTML-file in the original document, as separate entities. This decision effected our search engine, since the sub chapters would not be retrieved as separate hits. This created some noise in our results. 

%TODO: Finish this section

