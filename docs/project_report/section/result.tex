\section{Resultater}
\label{sec:result}

\subsection{Top 10 resultater ved benyttelse av VSM, med og uten vekting ved tf*idf}

\begin{table}[h!]
\begin{center}
\begin{tabular}{ | c c c c c c c c c | }
\hline
\textbf{Pos} & \textbf{Case1} & \textbf{Case2} & \textbf{Case3} & \textbf{Case4} & \textbf{Case5} & \textbf{Case6} & \textbf{Case7} & \textbf{Case8} \\ \hline
1 & t3.1 & t10.2 & t1.10 & t8.3 & t12.10 & t1.3 & t20.2 & t11.3 \\
2 & t10.2 & t19.5 & t1.14 & t19.6 & t1.6 & t11.3 & t21.1 & t10.11 \\
3 & t23.1 & t10.1 & t19.1 & t24 & t12.9 & t16.9 & t22.4 & t1.3 \\
4 & t14.5 & t10.3 & t3.1 & t8.4 & t19.6 & t1.11 & t20.1 & t1.10 \\
5 & t15.1 & t24.1 & t7.9 & t8.1 & t17.2 & t10.2 & t19.2 & t5.1 \\
6 & t21.1 & t11.1 & t8.1 & t15.3 & t13.2 & t10.3 & t17.3 & t1.13 \\
7 & t19.3 & t10.8 & t1.2 & t10.2 & t22.1 & t11.4 & t14.1 & t1.12 \\
8 & t8.1 & t22.3 & t1.11 & t19.5 & t12.3 & t1.10 & t6.2 & t1.9 \\
9 & t19.2 & t8.3 & t10.3 & t12.8 & t5.5 & t1.7 & t6.5 & t11.4 \\
10 & t16.13 & t14.5 & t8.8 & t8.6 & t1.17 & t12.3 & t22.2 & t16.5 \\
\hline
\end{tabular}
\end{center}
\caption{Resultater ved bruk av VSM(vektet ved tf*idf)}
\end{table}

\begin{table}[h!]
\begin{center}
\begin{tabular}{ | c c c c c c c c c | }
\hline
\textbf{Pos} & \textbf{Case1} & \textbf{Case2} & \textbf{Case3} & \textbf{Case4} & \textbf{Case5} & \textbf{Case6} & \textbf{Case7} & \textbf{Case8} \\ \hline
1 & t3.1 & t3.3 & t1.10 & t8.3 & t5.5 & t2.2 & t20.2 & t11.3 \\
2 & t15.1 & t8.9 & t15.1 & t2.2 & t1.10 & t2.1 & t20.1 & t1.7 \\
3 & t16.13 & t10.2 & t2.1 & t12.3 & t17.2 & t8.3 & t21.1 & t1.16 \\
4 & t14.1 & t1.7 & t6.2 & t8.10 & t8.9 & t12.3 & t22.4 & t24.1 \\
5 & t14.3 & t2.2 & t8.3 & t6.5 & t6.5 & t8.10 & t22.2 & t1.10 \\
6 & t8.1 & t2.1 & t1.7 & t5.4 & t10.2 & t5.4 & t19.2 & t6.2 \\
7 & t24.2 & t6.5 & t16.7 & t8.9 & t12.3 & t12.6 & t5.5 & t8.9 \\
8 & t10.2 & t12.3 & t4.6 & t1.10 & t1.7 & t17.1 & t12.1 & t19.6 \\
9 & t16.8 & t5.4 & t3.1 & t2.1 & t3.1 & t3.3 & t6.2 & t10.2 \\
10 & t7.7 & t8.3 & t21.1 & t10.2 & t2.2 & t8.9 & t17.3 & t3.3 \\
\hline
\end{tabular}
\end{center}
\caption{Resultater ved bruk av VSM(uten vekting ved tf*idf)}
\end{table}

\pagebreak
\subsection{Eksempler på ulike resultater med/uten tf*idf}
\subsubsection{Pasientsak nr 5}
Pasienten er en 64 år gammel kvinne som tidligere er stort sett frisk. De siste 5 månedene har hun merket følelse av ufullstendig tømming når hun har avføring. Ved flere anledninger har det vært spor av friskt blod i avføringen. Selv tilskriver hun dette hemorroider som hun har hatt før, og hun har ventet på at det skulle gå over.

Hun har vært hos sin egen lege som ikke finner noe galt ved vanlig undersøkelse. Ved rektaleksplorasjon (kjenne i endetarmsåpningen med finger) kjenner legen så vidt kanten av en uregelmessighet i tarmveggen. Legen ser spor etter gamle ytre hemorroider men ingen sannsynlig blødningskilde nå. Prøver på blod i avføringen er positive. Blodprøver viser en lett blodmangel (hemoglobin 10.8; normalt for kjønn og alder er >12). Øvrige blodprøver var normale. Pasienten blir henvist til colonoscopi (endoskopisk undersøkelse av tykktarmen).

\subsubsection{Resultater}
\begin{description}
\item{\textbf{Uten tf*idf: }}T5.5 Depresjoner
\item{\textbf{Med tf*idf: }}T12.10 Anorektale Forstyrrelser
\end{description}

Her ser vi at uten vekting ved bruk av tf*idf vil resultatet være ubrukelig, da det ikke har noen sammenheng med pasientens symptomer. Uten bruk av tf*idf vil T12.10(det mest nyttige dokumentet) ikke være blant de øverste 15 treffene søket vil returnere. Det er tydelig at tf*idf er essensielt i denne sammenheng.


\subsection{Pasientsak nr 6}
Ved første konsultasjon ble det funnet forstørrede tonsiller med hvitlig belegg og forstørrede glandler på begge sider av halsen. Det ble tatt halsprøve til strept.test som var positiv. Det fremkom at kjæresten nylig hadde gjennomgått streptokokktonsilitt. Det ble startet behandling med peroral penicillin (Apocillin) i vanlig dosering.

Pasienten kommer til ny konsultasjon etter manglende effekt av behandlingen med penicillin. Han ble verre, fikk mer svelgbesvær, fikk ikke i seg fast føde, og hadde også besværligheter med å få i seg væske. Samtidig var han vedvarende slapp og i dårlig allmenntilstand.

\subsection{Resultater}
\begin{description}
\item{\textbf{Uten tf*idf: }} T2.2 Legemiddelbehandling av vanlige kreftsykdommer
\item{\textbf{Med tf*idf: }} T1.3 Mononukleose
\end{description}
Resultatet uten tf*idf foreslår at pasienten har kreft, noe som kan være en svært forstyrrende diagnose dersom den gis feilaktig. Ved bruk av tf*idf vil Mononukleose dukke opp som et mye bedre resultat. Tf*idf gir en klar forbedring av søkeresultatene. 
