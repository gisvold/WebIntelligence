\section{Begrensninger og krav gitt av oppgaven}

Oppgaven ga enkelte begrensninger og krav til vår implementasjon.
\begin{enumerate}
\item{Autokodet ATC}
\begin{enumerate}
\item{Finn relevante ATC-koder bestemt av en sammenheng mellom beskrivelsen av ATC-koden og setningene i pasientjournalene, for hver enkelt setning i journalene.}
\item{For å lagre en sammenheng mellom en setning og en ATC-kode, lag en tabell hvor hver enkelt rad inneholder dokumentnummeret, setningsnummeret etterfulgt av den relevante ATC-koden (rangert).}
\item{Gjør tilsvarende for terapikapitlene i Legemiddelhåndboken. Inkluder kun subkapitler og subsubkapitler.}
\end{enumerate}
\item{Autokodet ICD-10}
\begin{enumerate}
\item{Tilsvarende som gjøres for ATC, skal gjøres med ICD-10, men ICD-10 klassifiseringer skal benyttes istedenfor.}
\end{enumerate}
\item{Rangering ved bruk av vektormodeller}
\begin{enumerate}
\item{Vi skal benytte metoder for informasjonsgjenfinning for å trekke ut og sammenligne semantisk informasjon i dokumenter, basert på statistisk informasjon.}
\end{enumerate}
\item{Evaluering}
\begin{enumerate}
\item{Vi skal skrive en rapport hvor vi presenterer teknikker/metoder for å evaluere resultatene automatisk.}
\item{For hver unike metode, skriv en evaluering}
\end{enumerate}
\item{Forbedre rangeringen}
\begin{enumerate}
\item{We skal kun benutte kodene fra oppgave 1 og 2 for å velge og rangere de relevante kapitlene i Legemiddelhåndboken.}
\item{Vi skal kombinere resulatene fra oppgave 6a med resultatene fra oppgave 3. Vi må også finne en måte å vekte de ulike rangeringene gitt av de ulike fremgangsmåtene.}
\end{enumerate}
\item{'Gold Standard'}
\begin{enumerate}
\item{Vi skal evaluere våre ulike rangeringsmetoder ved hjelp av en "gylden standard" gitt av fagstaben.}
\end{enumerate}
\item{Mulige forbedringer}
\begin{enumerate}
\item{Vi skal forklare ulike måter å videreutvikle og forbedre systemet på.}
\end{enumerate}
\end{enumerate}